%%%%%%%%%%%%%%%%%%%%%%%%%%%%%%%%%%%%%%%
% Deedy - One Page Two Column Resume
% LaTeX Template
% Version 1.2 (16/9/2014)
%
% Original author:
% Debarghya Das (http://debarghyadas.com)
%
% Original repository:
% https://github.com/deedydas/Deedy-Resume
%
% IMPORTANT: THIS TEMPLATE NEEDS TO BE COMPILED WITH XeLaTeX
%
% This template uses several fonts not included with Windows/Linux by
% default. If you get compilation errors saying a font is missing, find the line
% on which the font is used and either change it to a font included with your
% operating system or comment the line out to use the default font.
% 
%%%%%%%%%%%%%%%%%%%%%%%%%%%%%%%%%%%%%%
% 
% TODO:
% 1. Integrate biber/bibtex for article citation under publications.
% 2. Figure out a smoother way for the document to flow onto the next page.
% 3. Add styling information for a "Projects/Hacks" section.
% 4. Add location/address information
% 5. Merge OpenFont and MacFonts as a single sty with options.
% 
%%%%%%%%%%%%%%%%%%%%%%%%%%%%%%%%%%%%%%
%
% CHANGELOG:
% v1.1:
% 1. Fixed several compilation bugs with \renewcommand
% 2. Got Open-source fonts (Windows/Linux support)
% 3. Added Last Updated
% 4. Move Title styling into .sty
% 5. Commented .sty file.
%
%%%%%%%%%%%%%%%%%%%%%%%%%%%%%%%%%%%%%%%
%
% Known Issues:
% 1. Overflows onto second page if any column's contents are more than the
% vertical limit
% 2. Hacky space on the first bullet point on the second column.
%
%%%%%%%%%%%%%%%%%%%%%%%%%%%%%%%%%%%%%%


\documentclass[]{deedy-resume-openfont}
\usepackage{fancyhdr}
 
\pagestyle{fancy}
\fancyhf{}
 
\begin{document}

%%%%%%%%%%%%%%%%%%%%%%%%%%%%%%%%%%%%%%
%
%     LAST UPDATED DATE
%
%%%%%%%%%%%%%%%%%%%%%%%%%%%%%%%%%%%%%%
\lastupdated

%%%%%%%%%%%%%%%%%%%%%%%%%%%%%%%%%%%%%%
%
%     TITLE NAME
%
%%%%%%%%%%%%%%%%%%%%%%%%%%%%%%%%%%%%%%
\namesection{Srikanth}{Srinivas}{ \urlstyle{same}\href{https://srikanthsrnvs.github.io/website/}{Website} | \href{https://www.linkedin.com/in/srikanthsrnvs/}{LinkedIn}\\
\href{mailto:srikanth.srinivas@mail.utoronto.ca}{Email} | (647) 983-9837
}

%%%%%%%%%%%%%%%%%%%%%%%%%%%%%%%%%%%%%%
%
%     COLUMN ONE
%
%%%%%%%%%%%%%%%%%%%%%%%%%%%%%%%%%%%%%%

\begin{minipage}[t]{0.33\textwidth} 

%%%%%%%%%%%%%%%%%%%%%%%%%%%%%%%%%%%%%%
%     EDUCATION
%%%%%%%%%%%%%%%%%%%%%%%%%%%%%%%%%%%%%%

\section{Education} 

\subsection{University of Toronto}
\descript{Computer Science}
\location{July 2020 | Toronto, Canada}
Sub focus in deep learning
\sectionsep

%%%%%%%%%%%%%%%%%%%%%%%%%%%%%%%%%%%%%%
%     LINKS
%%%%%%%%%%%%%%%%%%%%%%%%%%%%%%%%%%%%%%

\section{Links} 
Facebook:// \href{https://www.facebook.com/IamGingerTrash}{\bf Srikanth Srinivas} \\
Github:// \href{https://github.com/srikanthsrnvs}{\bf srikanthsrnvs} \\
LinkedIn://  \href{https://www.linkedin.com/in/srikanthsrnvs/}{\bf srikanthsrnvs} \\
Blog://  \href{https://medium.com/@recurseai}{\bf recurseai} \\
Personal website://  \href{https://srikanthsrnvs.github.io/website/}{\bf Link} \\

%%%%%%%%%%%%%%%%%%%%%%%%%%%%%%%%%%%%%%
%     SKILLS
%%%%%%%%%%%%%%%%%%%%%%%%%%%%%%%%%%%%%%

\section{Skills}
\subsection{Programming}
\location{Over 7000 lines:}
Javascript \textbullet{}  Swift \textbullet{} Python \\
\location{Over 1000 lines:}
C \textbullet{} CSS \textbullet{} Java \textbullet{} HTML \textbullet{} \LaTeX\ \\
\location{Technologies:}
\textbullet{} React.js \\
\textbullet{} Xcode(iOS) \\
\textbullet{} Node.js \\
\textbullet{} SQL \\
\textbullet{} Tensorflow \\
\textbullet{} Keras\\
\textbullet{} PyTorch \\
\textbullet{} Firebase \\
\textbullet{} Google cloud platform \\
\textbullet{} Flask\\
\textbullet{} ROS\\
\sectionsep

\section{Certificates}
\subsection{Coursera}
\location{Nov 2019}
\textbullet{}Structuring Machine learning projects\\ \textbullet{} Neural networks and deep learning \\
\textbullet{} Improving deep neural networks: Hyperparameter tuning, Regularization and Optimization

\sectionsep

%%%%%%%%%%%%%%%%%%%%%%%%%%%%%%%%%%%%%%
%
%     COLUMN TWO
%
%%%%%%%%%%%%%%%%%%%%%%%%%%%%%%%%%%%%%%

\end{minipage} 
\hfill
\begin{minipage}[t]{0.66\textwidth} 

%%%%%%%%%%%%%%%%%%%%%%%%%%%%%%%%%%%%%%
%     EXPERIENCE
%%%%%%%%%%%%%%%%%%%%%%%%%%%%%%%%%%%%%%

\section{Experience}

\runsubsection{Scale.com}
\descript{| Forward deployed engineer }
\location{August 2020 - Present | San Francisco, CA}
\vspace{\topsep} % Hacky fix for awkward extra vertical space
\begin{tightemize}
\item I joined Scale to work at a hypergrowth startup on a new product they launched called Scale Nucleus
\item We were a 4 man team, with me leading the go-to-market strategy, and writing code to ensure we found the first few customers
\item I would occassional write features, but the most of my day was spent performing integrations for customers, or mapping their problems to software.
\item I grew the product from 3 to 40 customers in the span of two months, using a lot of scruffy startup playbook rules, in search of product market fit.
\end{tightemize}
\sectionsep

\runsubsection{Astrum.ai}
\descript{| Founder }
\location{Nov 2019 - July 2020 | Toronto, Ontario}
\vspace{\topsep} % Hacky fix for awkward extra vertical space
\begin{tightemize}
\item \textbf{\href{https://www.astrum.ai}{Astrum.ai}} is a no-code AutoML tool written in Python and React, using Tensorflow, Pytorch and Google Cloud
\item I am the sole engineer, and have built the platform from the ground up
\item It uses Reinforcement learning, transfer learning, and evolutionary simulations to search for ideal NN architectures
\item It's currently closed source, and a revenue generating project.
\end{tightemize}
\sectionsep

\runsubsection{Blip.delivery}
\descript{| Co-founder }
\location{May 2017 – Nov 2019 | Toronto, Ontario}
\vspace{\topsep} % Hacky fix for awkward extra vertical space
\begin{tightemize}
\item Blip was a same-day delivery API powered by crowd-sourced drivers, written in Reactjs, Node, Swift and Python
\item \textbf{\href{https://blip-live.firebaseapp.com/#intro}{Wrote Over 10,000 lines of backend API}}
\item \textbf{\href{https://apps.apple.com/ca/app/blip-driver/id1397140753}{Built an app used by over 300 daily active drivers}}
\item I was the sole engineer, and built the entire platform from the ground up and scaled it to ~1000 drivers and 20 stores
\end{tightemize}
\sectionsep


%%%%%%%%%%%%%%%%%%%%%%%%%%%%%%%%%%%%%%
%     Open source
%%%%%%%%%%%%%%%%%%%%%%%%%%%%%%%%%%%%%%

\section{Side Projects}
\runsubsection{Autonomous rovers}
\descript{| Engineer}
\location{Apr 2019 – Nov 2019 | Toronto, Ontario}
Built \textbf{\href{https://drive.google.com/drive/u/3/folders/16rhXzpnnE2r-Tf5yzjvB5i6v06zM74ir}{Blippy}} A self driving rover designed to carry 30kgs to perform same-day deliveries.
\item I engineered the frame, and worked with the low level modbus registers to get it working
\sectionsep

\runsubsection{PySimplex}
\descript{| Open-source project}
\location{Oct 2019 – Oct 2019 | Toronto, Ontario}
Built a simple \textbf{\href{https://github.com/srikanthsrnvs/pysimplex}{open source module}} to control Modbus motors designed by Simplex Motion. Taught myself the modbus protocol and write it atop PyModbus
\sectionsep


\end{minipage} 
\end{document}  \documentclass[]{article}
